\documentclass[letterpaper, titlepage]{article}

\usepackage[T1]{fontenc}
\usepackage[utf8]{inputenc}
\usepackage[francais]{babel}
\usepackage{geometry}
\usepackage{pgfplots}
\usepackage{tikz}
\usepackage{vdr}
\usepackage{palettechum}
\usepackage{logovdr}
\usepackage{logochum}
\usepackage{ntitlepage}

\usepgfplotslibrary{groupplots}

\tolerance=500
\setlength{\parskip}{.7em}
\setcounter{tocdepth}{2}
\def\arraystretch{1.5}


\def\ie{$i\, \colon e$}

\title{Notes de cours supplémentaires}
\author{Nicolas Blais St-Laurent}
\date{Septembre 2019}

\begin{document}

\begin{titlepage}
	\makeatletter
	\raggedright
	\parbox[b][\textheight][t]{.19\textwidth}{%
		\vspace{2\baselineskip}
		\logofvc{3cm}
		\vfill
	}
	\rule{1pt}{\textheight}
	\hspace{.04\textwidth}
	\parbox[b][\textheight][t]{.62\textwidth}{%
		\raggedright
		%\vspace{2\baselineskip}
		%\logofvc{3cm}
		\vfill
		{\huge\@title\\[2\baselineskip]}
		Par:\\[0.75\baselineskip]
		\@author\vfill
		\@date\\[1.5\baselineskip]
	}
	\makeatother
\end{titlepage}


\section{Contrôles du cyclage à haute fréquence}

Contrairement à ce que suggère l'identification des contrôles de l'appareil, les
paramètres de cyclage à haute fréquence contrôlés sont: la \emph{durée du
temps inspiratoire (Ti)} des percussions (contrôle identifié
\emph{Fréquence}) et leur \emph{ratio inspiration\hspace{0.5em}:\hspace{0.3em}expiration (\ie)}. La
fréquence de percussion (Fperc) et la durée du temps expiratoire des
percussions seront le résultat de ces deux réglages combinés.

\begin{figure}[b]
	\centering
	\begin{tikzpicture}[
		cid/.style={
			above=1.35*\ldist,
			align=center,
			font=\tiny
		},
		ie/.style={
			fill=black!15
		},
		index/.style={
			midway,
			below,
			inner
			sep=0,
			font=\tiny,
			scale=0.8
		},
		aug/.style={
			midway,
			above,
			inner
			sep=1mm,
			font=\tiny,
			scale=0.7
		},
		plusmoin/.style={
			below,
			inner sep=1mm,
			font=\tiny,
			%scale=0.7,
		},
		false/.style={
			draw=red,
			cross out
		}
]

\newcommand{\ldist}{7mm}

	\path (0,0) pic [ie, pic text=i] {aknob} 
	node [cid] {FREQUENCE\\DE PERCUSSION};

	\draw [->] (0,0) ++(\ldist, \ldist) -- ++(-2*\ldist, 0) 
	node [index] {$\blacktriangledown$}
	node [aug] {AUGMENTER}
	;

	\path (2.5,0) pic [ie, pic text=e] {aknob}
	node [cid] {RAPPORT\\i/e};

	\draw [<->] (2.5,0) ++(\ldist, -\ldist) 
	node [plusmoin] {+}
	-- ++(0, 2*\ldist) 
	-- ++(-2*\ldist, 0) 

	node [index] {$\blacktriangledown$}
	-- ++(0, -2*\ldist) 
	node [plusmoin] {-}
	node [aug] {}
	;

\end{tikzpicture}

	\caption{Contrôles du cyclage à haute fréquence, tels qu'identifiés sur l'appareil. Ce qui est réellement contrôlé par le bouton {\em i} est le temps inspiratoire des percussions. Et le rapport \ie\ {\em diminue} lorsqu'on tourne le bouton e en sens horaire.}
\end{figure}

\begin{table*}[h]
	\centering
	\caption{Paramètres du cyclage à haute fréquence}
	\begin{tabular}{ccc}
		\hline
		Réglage direct & & Réglage indirect\\
		\hline
		\parbox{3.7cm}{
			\raggedright
			Temps inspiratoire (Ti)\\
			Rapport \ie\
			} & $\rightarrow$ & \parbox{5cm}{
				\raggedright
				Temps expiratoire (Te)\\
				Fréquence de percussion (Fperc)
				}\\
				\hline
	\end{tabular}
\end{table*}


\subsection{Monitorage du rapport \ie}

On se rappellera que l'affichage du rapport \ie\ sur le Monitron n'est
fonctionnel que lorsque le Ti est plus petit que le Te. Lorsque le Ti
devient plus grand que le Te (rapport inversé), le Monitron affiche en
permanence 1\string: 1.0.

La meilleure façon de juger du rapport \ie\ est alors d'observer
l'apparence de la courbe de pression sur le monitron.

Les éléments à observer sont:

\begin{itemize}
\item
  Durée du Ti (montée de pression et plateau) versus celle du Te (chute
  de la pression) (observer à 1 ou 2 s par écran);
\item
  Présence d'un plateau. Un plateau où la pression plafonne complètement
		est suggestif d'un ratio inversé. Voir Figure~\ref{figie1}, courbe du bas.
  (observer à 1 ou 2 s par écran);
\item
  Espace sous la courbe de pression. L'augmentation de l'espace sous la
  courbe pendant l'inspiration convective est aussi suggestive d'un
  ratio inversé. Elle témoigne d'une diminution de l'amplitude des
  percussions. Voir Figure~\ref{figie8}, courbe du bas. (observer à 5 ou 8 s par
  écran);
\end{itemize}

\subsection{Implications du rapport \ie\ des
percussions}

Malgré la difficulté technique qu'il représente (absence de lecture
fiable et contrôle contre-intuitif), le réglage du rapport \ie\ des
percussions est un paramètre à ne pas négliger. Ce réglage influence
notamment:

\begin{itemize}
\item
  L'amplitude des percussions (différence entre la pression la plus basse
  et la plus élevée pendant une percussion),
\item
  La pression moyenne pouvant être atteinte
  pendant les phases inspiratoire et expiratoire de la convection.
\end{itemize}

Ainsi, un rapport \ie\ plus petit (Ti < Te) favorisera une plus
grande amplitude de percussion en permettant à la pression de
redescendre entre chaque percussion. Il sera par conséquent plus
difficile d'atteindre des pressions élevées, notamment pendant la phase
inspiratoire de la convection.

Inversement, un rapport \ie\ plus élevé (Ti > Te) permettra
d'atteindre des pressions plus élevées étant donné qu'il y aura moins de
temps pour que la pression diminue entre chaque percussion. L'amplitude
de variation de pression générée par chaque percussion sera par
conséquent diminuée.

Le bon réglage sera par conséquent ({\em ou pourrais être}) le plus petit rapport \ie\ (Ti court
et Te long) permettant d'atteindre les pressions inspiratoires
souhaitées.

\begin{table*}
	\centering
	\caption{Avantages et inconvénients d'un \ie\ bas ou élevé.}
	\begin{tabular}{cc}
		\hline
		Ti < Te & Ti > Te\\
		\hline
		\parbox{17em}{
			\raggedright
			Augmente amplitude de percussion\\
			Pressions inspiratoires moins élevées
			} & \parbox{16em}{
				\raggedright
				Diminue amplitude de percussion\\
				Pressions inspiratoires plus élevées
				}\\
				\hline
	\end{tabular}
\end{table*}


\begin{figure*}
	\def\iehuit{%
\addplot graphics [
	xmin=0,
	ymin=0,
	xmax=1,
	ymax=60
]}

\begin{tikzpicture}

\begin{groupplot}[
		group style={
			group size=1 by 2,
			xlabels at=edge bottom
		},
		enlargelimits=false,
		height=4cm,
		width=\textwidth,
		xtick={0, .25, .5, .75, 1},
		xlabel=Temps (s),
		ylabel=Pression (hPa)
]

\nextgroupplot
\iehuit {img/509.jpg};

\nextgroupplot
\iehuit{img/828.jpg};

\end{groupplot}
\end{tikzpicture}

	\caption{Rapport \ie\ adéquat (en haut) et rapport \ie\ inversé (en
	bas). On observe sur le tracé du bas un Te trop court ne permettant pas
	à la pression de redescendre entre chaque percussion. La pression
	d'équilibre est donc rapidement atteinte à la percussion suivante. Il en
	résulte une faible amplitude de variation de pression à chaque
	percussion. Vitesse de défilement à 1 s par écran.}
	\label{figie1}
\end{figure*}

\begin{figure*}
	\def\iehuit{%
\addplot graphics [
	xmin=0,
	ymin=0,
	xmax=8,
	ymax=60
]}

\begin{tikzpicture}

\begin{groupplot}[
		group style={
			group size=1 by 2,
			xlabels at=edge bottom
		},
		enlargelimits=false,
		height=4cm,
		width=\textwidth,
		xtick={0, 2, 4, 6, 8},
		xlabel=Temps (s),
		ylabel=Pression (hPa)
]

\nextgroupplot
\iehuit {img/329.jpg};

\nextgroupplot
\iehuit{img/629.jpg};

\end{groupplot}
\end{tikzpicture}

	\caption{Rapport \ie\ adéquat (en haut) et rapport \ie\ inversé (en
	bas). On observe une diminution de l'amplitude de percussion sur le
	tracé du bas. Vitesse de défilement à 8 s par écran.}
	\label{figie8}
\end{figure*}

\section{Données monitorées}

On retrouve à la fois des données mesurées sur le multimètre situé sur
le dessus du module de contrôle et sur le Monitron. Certaines données
sont même affichées aux deux endroits.

\emph{Il est important de noter que pour l'application du protocole de
ventilation du CHUM, c'est toujours les pressions affichées sur le
multimètre du module de contrôle que l'on doit utiliser (moyenne
inspiratoire et moyenne expiratoire).  Les pressions affichées par le Monitron (PEAK PRESSURE et PEEP/CPAP)
sont lues à la crête de l'oscillation et sont par conséquent peu
représentatives des pressions subies par les alvéoles pulmonaires.
}
\begin{table*}
	\centering
	\caption{Données affichées par le Multimètre numérique et par le Monitron. }
	\begin{tabular}{cc}
		\hline
		Multimètre du module de contrôle & Monitron \\
		\hline
		\parbox{16em}{
			\raggedright
			\emph{Pression inspiratoire moyenne *}\\
			\emph{Pression expiratoire moyenne *}\\
			Pression moyenne globale\\
			\emph{Fréquence de percussion ($F_{perc}$) *
			}
			} & \parbox{16em}{
				\raggedright
				Pression inspiratoire de crête\\
				Pression expiratoire de crête (PEP)\\
				Pression moyenne globale\\
				\emph{Ti (convection) *}\\
				\emph{Te (convection) *}\\
				I:E\\
				F\textsubscript{conv}\\
				\emph{Fréquence de percussion ($F_{perc}$) *}\\
				i:e
				}\\
				\hline
	\end{tabular}

	* \emph{Données utilisées dans le protocole clinique du CHUM}
\end{table*}


\begin{figure}
	\newcommand{\pexp}{5}
\newcommand{\pins}{18}
\newcommand{\arrpos}{1.06}
\begin{tikzpicture}[
		pline/.style={
			help lines,
			turquoisechum,
			rounded corners,
			out=0,
			in=180,
			thick
			},
		p/.style={
			circle,
			draw=turquoisechum,
			inner sep=0.5mm,
			thick
			}
	]

	\begin{axis}[
		width=0.65\textwidth,
		name=plot,
		font=\scriptsize,
		try min ticks=6,
		xtick={0,4,8},
		ytick={0,30},
		axis x line=bottom,
		axis y line=middle,
		enlarge y limits={value=0.25, upper},
		enlarge x limits={value=0.05, upper},
		extra y ticks={5, 18},
		extra y tick labels={$P_{ins. moy.}$, $P_{exp. moy.}$},
		extra y tick style={grid=major},
		major grid style={turquoisechum, thick}
		]

		\addplot [
			bleufoncechum,
			restrict x to domain=0:8,
			] table[x=time, y=Pao] {dat/f300.dat};

		\coordinate (D) at (axis cs: \pgfkeysvalueof{/pgfplots/xmax},\pins);
		\coordinate (B) at (axis cs: \pgfkeysvalueof{/pgfplots/xmax},\pexp);

	\end{axis}

	\pic [opacity=0.99, name=mm] at ([xshift=3.2cm, yshift=-.95cm]plot.east) {multimeter};

		\node [grad] (mmg50) at (mmscreen.north west) {50};
		\node [grad] (mmg0) at (mmscreen.south west) {0};
		\draw [pScale]	(mmg0) -- (mmg50) node [grad, left=0.0mm, pos=0.6, inner sep=0mm] {30};

		\node [below, white, align=left, font=\tiny] at (mmscreen.south) {Percussionaire\\Corporation};

	\node [p] (pmi) at (mmPmi) {\pins};
	\node [p] (pme) at (mmPme) {\pexp};

	%\draw [pline] (D) -- ([xshift=4mm]D) |- (pmi);
	%\draw [pline] (B) -- ([xshift=5mm]B) |- (pme);

	\draw [pline] (D) to (pmi);
	\draw [pline] (B) to (pme);
\end{tikzpicture}

	\caption{La pression inspiratoire moyenne et la pression expiratoire moyenne sont affichées sur me multimètre numérique se trouvant sur le dessus du ventilateur.}
\end{figure}

\section{Alarmes}

\subsection{Alarmes du module de contrôle}

\subsubsection*{Alarme de surpression}

Il s'agit d'une alarme pneumatique se déclenchant lors d'une surpression
dans le module de contrôle. Son déclenchement entraine une chute de la
pression délivrée. Une fois la cause corrigée, il faut réarmer l'alarme
(bouton poussoir rouge) pour que la ventilation reprenne normalement.

Au réglage le plus sensible (rotation en sens antihoraire) l'alarme se
déclenche lorsque la pression dans le circuit avoisine les 80 $cmH_2O$.

Lorsque cette alarme se déclenche, il faut en premier lieu suspecter un
réglage inadéquat (par exemple fonction \emph{PRESSION DE CONVECTION} ou
\emph{PEP non oscillante} activées ou fréquence de percussion inférieure
à 100) ou une tubulure blanche coincée.

Il est improbable qu'une condition clinique (par exemple toux ou
résistances augmentées) entraine l'activation de cette alarme.

\subsubsection*{Alarme de déconnection}

Il s'agit d'un module indépendant situé sur le côté de l'appareil et
alimenté par une batterie. Cette alarme se déclenche lorsqu'aucune
pression n'est détectée dans le circuit pour une période donnée. Cette
période peut (en théorie\ldots{}) être ajustée au moyen de la roulette
noire.

\subsubsection*{Alarme du mélangeur air-oxygène}

Il s'agit d'une alarme pneumatique se déclenchant lorsque le mélangeur
perd son alimentation en air ou en oxygène. Il n'y a pas de fonction
\emph{silence} ou \emph{réarmer~}: l'alarme s'arrête automatiquement
lorsque l'alimentation des deux gaz est rétablie.

\subsection{Alarmes du Monitron}

\subsubsection*{Alarme de pression haute}%

Cette alarme se déclenche dès que la pression dans le circuit est
supérieure à la limite réglée. La valeur du réglage est indiquée par une
ligne rouge dans la zone de graphiques.

Son réglage répond même logique que l'alarme de pression haute en
ventilation conventionnelle (par exemple 10 $cmH_2O$ de plus que la
pression de crête actuelle). Il faut cependant se rappeler que le
déclenchement de l'alarme n'interrompt pas la ventilation étant donné
que le Monitron et le module de contrôle sons indépendants l'un de
l'autre.

\subsubsection*{Alarme de pression basse}

Cette alarme s'active lorsque la pression dans le circuit est inférieure
au seuil réglé pour plus de 6 s (alarme visuelle) et 12 s (alarme
sonore).

Il est à noter qu'une fois la pression rétablie, l'alarme continue à
sonner tant qu'elle n'a pas été réarmée.

\section{Séquence des réglages}

\begin{figure*}
	\centering
\begin{tikzpicture}

	\pic [name=VDR, black!60, scale=0.8] {vdr};

	\begin{scope}[
		every node/.style={
			color=black,
			}
			]
	\node (1) at (VDR-e) {1};
	\node (2) at (VDR-i) {2};
	\node (3) at (VDR-F) {3};
	\node (4) at (VDR-O) {4};
	\node (5) at (VDR-I) {5};
	\node (6) at (VDR-E) {6};
	\end{scope}

	\begin{scope}[
		every node/.style={
			yshift=8mm,
			align=center,
			scale=0.5
			}
			]
	\node at (VDR-e) {RAPPORT\\i/e};
	\node at (VDR-i) {FREQUENCE\\DE PERCUSSION};
	\node at (VDR-F) {DEPIT\\PULSE};
	\node at (VDR-O) {CPAP\\OSCILLANTE};
	\node at (VDR-I) {TEMPS\\INSPIRATOIRE};
	\node at (VDR-E) {TEMPS\\EXPIRATOIRE};
	\end{scope}

	\begin{scope}[
		every path/.style={
			black,
			opacity=0.80,
			line width=0.7mm,
			->
			},
			]
	\draw [] (1) to (2);
	\draw [bend left=27] (2) to (3);
	\draw [bend left=60] (3) to (4);
	\draw [bend left=45] (4) to (5);
	\draw [bend left=45] (5) to (6);
	\end{scope}

\end{tikzpicture}

	\caption{Séquence de réglage des paramètres.}
\end{figure*}

En raison des interactions entre les différents réglages, il est
judicieux de régler en premier les paramètres ayant beaucoup d'influence
sur les autres réglages, ou influençant plusieurs autres réglages.

Ainsi, avant d'effectuer quelque réglage que ce soit, on s'assurera que
la pression de travail est réglée à 40 lb/po\textsuperscript{2} et que la nébulisation est
en fonction. On s'assurera aussi que la PEP non oscillante et
l'augmentation des pressions de convection (3\textsuperscript{e} phase) sont désactivées
(tourné complètement en sens horaire).

Ensuite, étant donné que le rapport \ie\ des percussions (haute
fréquence) influence à la fois la fréquence de percussion et l'amplitude
des percussions (donc les pressions de ventilation), il est judicieux
d'ajuster ce paramètre en tout premier lieu.

Une fois le rapport \ie\ des percussions ajusté,~le temps inspiratoire
des percussions peut être ajusté à n'importe quel moment pour régler la
fréquence de percussion.

Pour les paramètres d'amplitude de percussion, l'amplitude des
percussions pendant l'inspiration influence celle pendant l'expiration.
Il convient donc de toujours ajuster la pression inspiratoire avant la
pression expiratoire.

Finalement, les pressions de ventilation ayant une influence sur le
temps inspiratoire et expiratoire de la convection (basse fréquence),
on attendra d'avoir ajusté les pressions de ventilation avant de régler
avec précision ces deux paramètres.~

\end{document}
